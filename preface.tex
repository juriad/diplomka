\chapter*{Introduction}
\addcontentsline{toc}{chapter}{Introduction}

Tree data structures are one of the most commonly used in computer science, and as such they have been studied for decades.
Many of the data structures are designed to general purpose -- they support most operations which their user may need.
However, if more information is known about the tree and the operations which are required, then it is possible to come up with a specialized tree data structure which is better for the particular use-case than any general purpose one.

We will be focusing on one type of trees: ordinal unlabeled trees.
Ordinal means that children of a vertex $v$ are ordered and assigned numbers, to which we refer as child rank, starting with $1$ which is assigned to the leftmost child, up to $\degree(v)$ which gets the rightmost child.
Unlabeled in our case means that all vertices are treated equally by all operations.

Moreover we will focus on static data structures that support all operations in constant time and that are succinct -- their space complexity is close to the optimal one.
We will define what the restriction on space complexity means in the first chapter.

This problem of space-efficient data structures for representing trees was first studied by \cite{jacobson1989space} and since then resulted in several different solutions.
Succinct representations of trees have found several applications, namely compression of suffix trees as described in \cite{jansson2012ultra} and XML document processing described in \cite{geary2006succinct}.

We investigate and describe the main data structures which are dominant in this area.
Not only do we summarize the current state of the succinct encodings of trees, we also come up with own results.
We propose several operations which has not been supported before by their respective representations
We also offer alternative implementations or small improvements of others.

\section*{Organization of This Thesis}
\addcontentsline{toc}{section}{Organization of This Work}

In the first chapter we define the environment in which we present all results in the thesis.
This includes the definition of the computation model which we use when we formulate algorithms, types of space-efficient data structures, and general techniques which we use in the following chapters.

The second chapter is dedicated to a simpler data structures than trees.
We present results about supporting various operations on general and balanced bit strings.
Since bit strings can be seen as characteristic vectors of a subset of an interval $[0, N)$, we show results about representing sets and supporting the same operations on them.
We also define a structure called compressed array which we use extensively in the last part of the thesis.

In the third chapter, we define a set of operations on trees which we consider later when developing various tree data structures.
We present three representations of trees which are based on a single bit string encoding the whole tree.
Not only are they all very simple, they are also limited in the operations which they support.

In the last chapter, we extend one of the previous representations into a data structure which supports nearly all possible operations.
We also describe the only data structure which does not reduce the tree into a simple bit string.
The last result in this thesis connects three of the previous representations.

We conclude with a summary of operations which each of the representations supports.
In the end we describe what the contribution of our work is.