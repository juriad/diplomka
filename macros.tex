%%% This file contains definitions of various useful macros and environments %%%
%%% Please add more macros here instead of cluttering other files with them. %%%

%%% Minor tweaks of style

% These macros employ a little dirty trick to convince LaTeX to typeset
% chapter headings sanely, without lots of empty space above them.
% Feel free to ignore.
\makeatletter
\def\@makechapterhead#1{
  {\parindent \z@ \raggedright \normalfont
   \Huge\bfseries \thechapter. #1
   \par\nobreak
   \vskip 20\p@
}}
\def\@makeschapterhead#1{
  {\parindent \z@ \raggedright \normalfont
   \Huge\bfseries #1
   \par\nobreak
   \vskip 20\p@
}}
\makeatother

% This macro defines a chapter, which is not numbered, but is included
% in the table of contents.
\def\chapwithtoc#1{
\chapter*{#1}
\addcontentsline{toc}{chapter}{#1}
}

% Draw black "slugs" whenever a line overflows, so that we can spot it easily.
\overfullrule=1mm

%%% Macros for definitions, theorems, claims, examples, ... (requires amsthm package)

\theoremstyle{plain}
\newtheorem{thm}{Theorem}
\newtheorem{lemma}[thm]{Lemma}
\newtheorem{claim}[thm]{Claim}

\theoremstyle{plain}
\newtheorem{defn}{Definition}

\theoremstyle{remark}
\newtheorem*{cor}{Corollary}
\newtheorem*{rem}{Remark}
\newtheorem*{example}{Example}

%%% An environment for proofs

%%% FIXME %%% \newenvironment{proof}{
%%% FIXME %%%   \par\medskip\noindent
%%% FIXME %%%   \textit{Proof}.
%%% FIXME %%% }{
%%% FIXME %%% \newline
%%% FIXME %%% \rightline{$\square$}  % or \SquareCastShadowBottomRight from bbding package
%%% FIXME %%% }

%%% An environment for typesetting of program code and input/output
%%% of programs. (Requires the fancyvrb package -- fancy verbatim.)

\DefineVerbatimEnvironment{code}{Verbatim}{fontsize=\small, frame=single}

%%% The field of all real and natural numbers
\newcommand{\R}{\mathbb{R}}
\newcommand{\N}{\mathbb{N}}

%%% Useful operators for statistics and probability
\DeclareMathOperator{\pr}{\textsf{P}}
\DeclareMathOperator{\E}{\textsf{E}\,}
\DeclareMathOperator{\var}{\textrm{var}}
\DeclareMathOperator{\sd}{\textrm{sd}}

%%% Transposition of a vector/matrix
\newcommand{\T}[1]{#1^\top}

%%% Various math goodies
\newcommand{\goto}{\rightarrow}
\newcommand{\gotop}{\stackrel{P}{\longrightarrow}}
\newcommand{\maon}[1]{o(n^{#1})}
\newcommand{\abs}[1]{\left|{#1}\right|}
\newcommand{\dint}{\int_0^\tau\!\!\int_0^\tau}
\newcommand{\isqr}[1]{\frac{1}{\sqrt{#1}}}

%%% Various table goodies
\newcommand{\pulrad}[1]{\raisebox{1.5ex}[0pt]{#1}}
\newcommand{\mc}[1]{\multicolumn{1}{c}{#1}}


%%% My own macros

% DEFINITION of this ugly word
\def\naive{naïve}

% REDEFITION of phi
\let\uglyphi\phi
\let\phi\varphi

% DEFINITION of \ph
\def\ph{\relax\ifmmode*\else$*$\fi}

% DEFINITION of percent
\let\percent\%
\def\%{\ifmmode\mathbin{\percent}\else\percent\fi}

% DEFINITION of \stuct
\def\struct#1#2{#1{:}\ #2}

% DEFINITION of \todo
\newcommand{\todo}[1]{}
\renewcommand{\todo}[1]{{\color{red} TODO: {#1}}}

% DEFINITION of \str
\def\str#1{\relax\ifmmode\texttt{"#1"}\else\texttt{"#1"}\fi}
\def\openingParen{\str(}
\def\closingParen{\str)}

% DEFINITION of titles
\usepackage[section]{placeins}
\usepackage[nobottomtitles*]{titlesec}
\renewcommand{\bottomtitlespace}{.15\textheight}

% OVERPAREN
\def\x#1#2{\underset{#2}{#1}}

% ALLOWS redefinition of math symbols
\makeatletter
\newcommand{\DeclareMathActive}[2]{%
  % #1 is the character, #2 is the definition
  \expandafter\edef\csname keep@#1@code\endcsname{\mathchar\the\mathcode`#1 }
  \begingroup\lccode`~=`#1\relax
  \lowercase{\endgroup\def~}{#2}%
  \AtBeginDocument{\mathcode`#1="8000 }%
}

\newcommand{\std}[1]{\csname keep@#1@code\endcsname}
%\patchcmd{\newmcodes@}{\mathcode`\-\relax}{\std@minuscode\relax}{}{\ddt}
%\AtBeginDocument{\edef\std@minuscode{\the\mathcode`-}}
% \DeclareMathActive{;}{,\ }
\makeatother

% REDEFINITION of underscore
\makeatletter
\newcommand*{\IfBoldFaceTF}{%
  \ifx\f@series\my@test@bx
    \expandafter\@firstoftwo
  \else
    \expandafter\@secondoftwo
  \fi
}
\newcommand*{\my@test@bx}{bx}
\makeatother


\let\origUnderscore\_
\def\_{\leavevmode%
	\kern.03em\ifmmode%
		\vbox{\hrule width.4em height.4pt}%
	\else%
		\vbox{\IfBoldFaceTF{\hrule width.4em height.6pt}{\hrule width.4em height.4pt}}%
	\fi%
}

% DEFINITION of mathematic opertors
\newcommand{\bitnot}{\mathord{\sim}}
\newcommand{\bitand}{\mathbin{\&}}
\newcommand{\bitor}{\mathbin{|}}
\newcommand{\bitxor}{\mathbin{^\wedge}}
\newcommand{\bitlsh}{\mathbin{\ll}}
\newcommand{\bitrsh}{\mathbin{\gg}}

\newcommand{\boolnot}{\mathord{\textrm{not}\,}}
\newcommand{\booland}{\mathrel{\textrm{and}}}
\newcommand{\boolor}{\mathrel{\textrm{or}}}

\DeclareMathOperator*{\argmin}{arg\,min}
\DeclareMathOperator*{\argmax}{arg\,max}

\def\textif{\ \textrm{\textnormal{if}}\ }
\def\textthen{\ \textrm{\textnormal{then}}\ }
\def\textelse{\ \textrm{\textnormal{else}}\ }
\def\textotherwise{\ \textrm{\textnormal{otherwise}}}

%DEFINITION for plotting
\usepackage{tikz}

% DEFINITION of tables
\usepackage{tabularx}
\newcolumntype{L}[1]{>{\hsize=#1\hsize\raggedright\arraybackslash}X}%
\newcolumntype{J}[1]{>{\hyphenpenalty=100\hsize=#1\hsize\arraybackslash}X}%
%\newcolumntype{R}[1]{>{\hsize=#1\hsize\raggedleft\arraybackslash}X}%
%\newcolumntype{C}[2]{>{\hsize=#1\hsize\columncolor{#2}\centering\arraybackslash}X}%
\usepackage{rotating} % provides sidewise table

% DEFINITION of algorithm with default placement
% must be before hyperref
%\usepackage{float}
%\usepackage[plain]{algorithm}
%\makeatletter
%\renewcommand{\fps@algorithm}{!ht}
%\makeatother
\newenvironment{algorithm}[0]{\allowbreak\begin{samepage}}{\end{samepage}\allowbreak}

% defaults:
%\setlength{\floatsep}{12.0pt plus 2.0pt minus 4.0pt}
%\setlength{\textfloatsep}{20.0pt plus 2.0pt minus 4.0pt}
%\setlength{\intextsep}{14.0pt plus 4.0pt minus 4.0pt}

%\setlength{\floatsep}{8.0pt plus 2.0pt minus 4.0pt}
%\setlength{\textfloatsep}{10.0pt plus 2.0pt minus 4.0pt}
%\setlength{\intextsep}{8.0pt plus 4.0pt minus 4.0pt}

%\setlength{\floatsep}{5pt plus 2pt minus 2pt}
%\setlength{\textfloatsep}{10pt plus 5pt minus 5pt}
%\setlength{\intextsep}{5pt plus 2pt minus 2pt}
%\setlength{\abovecaptionskip}{0pt}
%\setlength{\abovecaptionskip}{10pt}
%\renewcommand\textfraction{0.0}
%\setcounter{totalnumber}{100}
%\renewcommand\floatpagefraction{0.1}

% DEFINITION of algorithmic with some tweaks
\usepackage[noend]{algpseudocode}
\def\algcolon{\textbf{:}}
\algrenewtext{Function}[2]{\algorithmicfunction\ \textsc{#1}(#2)\algcolon}
\algrenewtext{ForAll}[1]{\algorithmicforall\ #1\algcolon}
\algrenewtext{For}[1]{\algorithmicfor\ #1\algcolon}
\algrenewtext{While}[1]{\algorithmicwhile\ #1\algcolon}
\algrenewtext{If}[1]{\algorithmicif\ #1\algcolon}
\algrenewtext{ElsIf}[1]{\algorithmicelse\ \algorithmicif\ #1\algcolon}
\algrenewtext{Else}{\algorithmicelse\algcolon}
\algnewcommand\Break{\textbf{break}}
\def\Instr{,\quad}

% ALLOWS removing algorithms
\usepackage{environ}
%\RenewEnviron{algorithmic}[2][\relax]{}


%DEFINITIONS of \defineName
\usepackage{xifthen}
\newcommand{\defineName}[2][]{%
	\expandafter\newcommand\csname \if\relax\detokenize{#1}\relax#2\else#1\fi\endcsname{\hyphenpenalty=500\relax\ifmmode\mathit{#2}\else\textsc{#2}\fi}%
}
\def\m#1{\relax\ifmmode#1\else$#1$\fi}

%DEFINITIONS of names

% deque and array operations
\defineName{push}
\defineName{pop}
\defineName{enqueue}
\defineName{dequeue}
\defineName{append}
\defineName{prepend}
\defineName{concatenate}
\defineName{reverse}

% output
\defineName[out]{output}

% in Introduction
\defineName{val}

% in Foundations
\defineName{true}
\defineName{false}

% in Bit Strings, general definitions
\defineName{rank}
\defineName{select}
\defineName{inspect}
\defineName{pred}
\let\origSucc\succ
\let\succ\relax
\defineName{succ}
\defineName{prev}
\defineName[nextt]{next}

% in Bit Strings, rank
\defineName{block}
\defineName[glob]{global}

% in Bit Strings, select
\defineName[blockEnum]{block\_enum}
\defineName{blocks}
\defineName[smallBlockEnum]{small\_block\_enum}
\defineName[smallBlocks]{small\_blocks}

% in Bit Strings, balanced definitions
\defineName{match}
\defineName{enclose}
\defineName[findClose]{find\_close}
\defineName[findOpen]{find\_open}
\defineName{excess}
\defineName[parenDepth]{paren\_depth}
\defineName{rmqi}
\defineName{rmq}
\defineName{RMQi}
\defineName{RMQ}

% in Bit Strings, match, enclose
\defineName{far}
\defineName{paren}
\defineName[fwdSearch]{fwd\_search}
\defineName[bwdSearch]{bwd\_search}

% in Bits Strings, rmq
\defineName[rmqiSpan]{rmqi\_span}
\defineName{Tm}

% in Bit Strings, sublinear FID
\defineName[characteristicBlock]{characteristic\_block}

% in Bit Strings, compressed array
\newcommand{\defineCompressed}[2][]{%
	\def\name{\if\relax\detokenize{#1}\relax#2\else#1\fi}
	\expandafter\defineName\expandafter[\name FID]{#2\_FID}
	\expandafter\defineName\expandafter[\name Elems]{#2\_elems}
	\expandafter\defineName\expandafter[\name Before]{#2\_before}
	\expandafter\defineName\expandafter[\name Parts]{#2\_parts}
}
\defineCompressed{A}
\defineName[elementIndex]{element\_index}
\defineName[runFirst]{run\_first}
\defineName[runLast]{run\_last}
\defineName[runLength]{run\_length}
\defineName{before}
\defineName[compressedArray]{compressed\_array}
\defineName{op}

\newcommand{\defineRS}[2][]{%
	\def\name{\if\relax\detokenize{#1}\relax#2\else#1\fi}
	\expandafter\defineName\expandafter[\name Rank]{#2\_rank}
	\expandafter\defineName\expandafter[\name Select]{#2\_select}
	\expandafter\defineName\expandafter[\name Any]{#2\_\ph}
}

\newcommand{\defineOps}[2][]{%
	\edef\name{\if\relax\detokenize{#1}\relax#2\else#1\fi}
	\expandafter\defineName\expandafter[\name Rank]{#2\_rank}
	\expandafter\defineName\expandafter[\name Select]{#2\_select}
	\expandafter\defineName\expandafter[\name Prev]{#2\_prev}
	\expandafter\defineName\expandafter[\name Next]{#2\_next}
	\expandafter\defineName\expandafter[\name First]{#2\_first}
	\expandafter\defineName\expandafter[\name Last]{#2\_last}
	\expandafter\defineName\expandafter[\name Size]{#2\_size}
	\expandafter\defineName\expandafter[\name Any]{#2\_\ph}
}

% in Trees, operations
\defineOps[any]{\ph}
\defineRS{pre}
\defineRS{post}
\defineRS{in}
\defineRS{dfuds}
\defineRS{lo}

\defineOps{child}
\defineName{degree}
\defineName{parent}
\defineName[isRoot]{is\_root}

\defineOps{level}
\defineName[dep]{depth}

\defineOps{leaf}
\defineName[isLeaf]{is\_leaf}

\defineName[isAncestor]{is\_ancestor}
\defineName[levelAncestor]{level\_ancestor}
\defineName{lca}
\defineName{distance}

\defineName[hei]{height}
\defineName[deepestVertex]{deepest\_vertex}
\defineName[subtreeSize]{subtree\_size}

% representations
\defineName[LOUDSRepresentation]{LOUDS\_representation}
\defineName[BPRepresentation]{BP\_representation}
\defineName[DFUDSRepresentation]{DFUDS\_representation}

% in Trees, LOUDS
\defineName{children}
\defineName[toOnes]{to\_ones}
\defineName[toZeros]{to\_zeros}
\defineName[toBeginning]{to\_beginning}
\defineName[blockOfD]{block\_of\_D}

% in Trees, BP
\defineName{state}
\defineName[UNSEEN]{\str{UNSEEN}}
\defineName[OPEN]{\str{OPEN}}
\defineName[CLOSED]{\str{CLOSED}}

% in Trees, DFUDS
\defineName[toEnd]{to\_end}
\defineName[toLast]{to\_last}
\defineName[toSymmetric]{to\_symmetric}

% in structures, FF operations
\defineName[summ]{sum}
\defineName[rmqSize]{rmq\_size}
\defineName[rmqRank]{rmq\_rank}
\defineName[rmqSelect]{rmq\_select}
\defineName[RMQSize]{RMQ\_size}
\defineName[RMQRank]{RMQ\_rank}
\defineName[RMQSelect]{RMQ\_select}

% in structures, FF min-max tree
\defineName{ms}
\defineName{Ms}
\defineName[sumBlock]{sum\_block}
\defineName[fwdSearchBlock]{fwd\_search\_block}
\defineName[nodeSearch]{node\_search}
\defineName[rightSiblings]{right\_siblings}
\defineName[rmqInfo]{rmq\_info}
\defineName[rmqList]{rmq\_list}
\defineName[rmqBlock]{rmq\_block}
\defineName[rmqSizeBlock]{rmq\_size\_block}
\defineName[rmqSelectBlock]{rmq\_select\_block}
\defineName[leftSiblingsInclusive]{left\_siblings\_inclusive}
\defineName[minSearch]{min\_search}
\defineName[nodeRange]{node\_range}

% in structures, FF macro search
\defineName{lrm}
\defineName{LRM}
\defineName{rlm}
\defineName{RLM}
\defineName[lrmSearch]{lrm\_search}
\defineName[LRMSearch]{LRM\_search}
\defineName[roott]{root}
\defineName{length}
\defineName{start}
\defineName[en]{end}
\defineName{ladder}

% in structures, FF macro range
\defineName{prefix}
\defineName{suffix}
\defineName[spann]{span}
\defineName{mr}
\defineName{mi}
\defineName[mpp]{mp}

% in structures, TC decomposition
\defineName{heavy}
\defineName{decomposeSubtree}
\defineName{decompose}
\defineName{zip}

% in structures, TC representation
\defineName{size}
\defineName{rep}
\defineName{offset}
\defineName[primaryI]{primary_1}
\defineName[primaryII]{primary_2}
\defineName[parentI]{parent_1}
\defineName[parentII]{parent_2}
\defineName[bottomI]{bottom_1}
\defineName[bottomII]{bottom_2}
\defineName[dummyI]{dummy\m{_1}}
\defineName[dummyII]{dummy\m{_2}}
\defineName[childrenI]{children_1}
\defineName[childrenII]{children_2}
\defineName[typeIII]{type3}
\defineName[childrenIndexBoth]{children\_index_{\{1,2\}}}
\defineName[childrenIndexI]{children\_index_1}
\defineName[childrenIndexII]{children\_index_2}
\defineName[childrenParentBoth]{children\_parent_{\{1,2\}}}
\defineName[childrenParentI]{children\_parent_1}
\defineName[childrenParentII]{children\_parent_2}
\defineName[microTreeOffsets]{micro\_tree\_offsets}
\defineName[microTrees]{micro\_trees}
\defineName[miniTreeOffsets]{mini\_tree\_offsets}
\defineName[miniTrees]{mini\_trees}

% in structures, TC navigation
\defineName[dummyUp]{dummy\_up}
\defineName[dummyDown]{dummy\_down}
\defineName{canonize}
\defineName[rootI]{root\m{_1}}
\defineName[rootII]{root\m{_2}}

% in strutures, TC depth
\defineName[depthI]{depth_1}
\defineName[depthII]{depth_2}
\defineName[subtreeSizeI]{subtree\_size_1}
\defineName[subtreeSizeII]{subtree\_size_2}
\defineName[deepestVertexI]{deepest\_vertex_1}
\defineName[deepestVertexII]{deepest\_vertex_2}
\defineName[dummyAncestor]{dummy\_ancestor}

% in structures, TC ranking, pre
\defineName[preFirstI]{pre\_first_1}
\defineName[preFirstII]{pre\_first_2}
\defineName[preVerticesI]{pre\_vertices_1}
\defineName[preVerticesII]{pre\_vertices_2}
\defineName{first}

% in structures, TC ranking, post
\defineName[terminaryI]{terminary_1}
\defineName[terminaryII]{terminary_2}
\defineName[postLastI]{post\_last_1}
\defineName[postLastII]{post\_last_2}
\defineName[postVerticesI]{post\_vertices_1}
\defineName[postVerticesII]{post\_vertices_2}
\defineName{last}

% in sructures, TC ranking, dfuds
\defineName[dfudsAnc]{dfuds\_anc}
\defineName{ancestors}
\defineName[dfudsVerticesI]{dfuds\_vertices_1}
\defineName[dfudsVerticesII]{dfuds\_vertices_2}

% in structures, TC ranking, in
\defineName[inAnc]{in\_anc}
\defineName[inSize]{in\_size}
\defineName[leftSiblings]{left\_siblings}
\defineName[inVerticesI]{in\_vertices_1}
\defineName[inVerticesII]{in\_vertices_2}
\defineName[leftTypeIIi]{left\_type2_1}
\defineName[leftTypeIIii]{left\_type2_2}

% in structures, TC ancestral
\defineName[lcaI]{anc\_search\m{_1}}
\defineName[lcaII]{anc\_search\m{_2}}
\defineName[ancSearchI]{anc\_search\m{_1}}
\defineName[ancSearchII]{anc\_search\m{_2}}

% in structures, TC level, first last
\defineName[lfirstI]{lfirst_1}
\defineName[lfirstII]{lfirst_2}
\defineName[lfirstKI]{lfirst\_k_1}
\defineName[lfirstKII]{lfirst\_k_2}
\defineName[ldSearchI]{ld\_search\m{_1}}
\defineName[ldSearchII]{ld\_search\m{_2}}
\defineName[llastI]{llast_1}
\defineName[llastII]{llast_2}
\defineName[llastKI]{llast\_k_1}
\defineName[llastKII]{llast\_k_2}

% in structures, TC level, next
\defineName[dummyNext]{dummy\_next}
\defineName[lnextI]{lnext_1}
\defineName[lnextII]{lnext_2}

% in structures, USR intro
\defineName[bpSubstring]{bp\_substring}
\defineName[dfudsSubstring]{dfuds\_substring}
\defineName[tcMicrotree]{tc\_microtree}

% in structures, USR restriction
\defineName[bp]{bp}
\defineName[dfuds]{dfuds}
\defineName[bpDummy]{bp\_dummy}
\defineName[dfudsDummy]{dfuds\_dummy}
\defineName[bpReduce]{bp\_reduce}
\defineName[dfudsReduce]{dfuds\_reduce}
\defineName[degreeOwn]{degree\_own}
\defineName[bpLast]{bp\_last}
\defineName[bpFollowing]{bp\_following}
\defineName[dfudsLast]{dfuds\_last}
\defineName[dfudsFollowing]{dfuds\_following}
\defineName[dfudsStart]{dfuds\_start}

% in structures, USR skinny
\defineName{restore}
\defineName[readTree]{read\_tree}
\defineName[bpSkinnyDown]{bp\_skinny\_down}
\defineName[bpSkinnyUp]{bp\_skinny\_up}
\defineName[dfudsSkinnyDown]{dfuds\_skinny\_down}
\defineName[dfudsSkinnyUp]{dfuds\_skinny\_up}
\defineName[bpRep]{bp\_rep}
\defineName[dfudsRep]{dfuds\_rep}
\defineName{ip}
\defineName[bpSubstringSkinny]{bp\_substring\_skinny}
\defineName[dfudsSubstringSkinny]{dfuds\_substring\_skinny}
\defineName[bpIndex]{bp\_index}
\defineName[dfudsIndex]{dfuds\_index}

% in structures, USR general
\defineName[bpNext]{bp\_next}
\defineName[dfudsNext]{dfuds\_next}