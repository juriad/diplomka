\chapter{Trees}



\section{Succinct trees}

There are many ways to represent 

\subsection{Universe}


\subsection{Representations}

A tree can be stored as a bit string in several different ways:
\begin{itemize}
	\item Level-Order Unary Degree Sequence -- a simple heap-like structure which supports the basic navigational queries requiring only rank and select;
	\item Balanced Parentheses -- uses the natural mapping between ordinal trees and matching parentheses;
	\item Depth-First Unary Degree Sequence -- combination of the previous two approaches offering an alternative mapping to matching parentheses;
	\item Fully-Functional -- a data structure built on top of matching parentheses using different primitives then rank and select;
	\item Tree Covering -- recursive decomposition of the tree into smaller trees which are then encoded;
\end{itemize}


A tree can also be encoded using various traversal methods resulting in sequence of zeros and ones, or often opening and closing parentheses, which is equivalent.
The other method is based on decomposition of the tree into smaller subtrees, reusing some concepts from indices used in the first method.

\section{Indices}

All basic navigation in trees is realized as a composition of simple operations on bit strings.





\section{LOUDS}

The first method discovered was introduced by Jacobson in YEAR.
It is based on algorithm BFS and noting degrees of individual vertices.


\section{BP}

Balanced parentheses representation is defined recursively.
A leaf is encoded a pair of parentheses.
An inner vertex is encoded as a concatenated list of encoding of children enclosed by a pair of parentheses.

% TODO an example of a tree with BP encoding

Each vertex is associated with the first opening parenthesis of its representation.
Simple navigation in the tree relative to a vertex $v$ which is at position $i$.

The subtree rooted in the current vertex is fully contained in the substring $BP[i, find\_close(i)]$.
The vertex is a leaf if $BP[i+1]$ is a close parenthesis.
The first child is at position $i+1$ unless the vertex is a leaf.
The vertex is the root if $i$ is equal to $0$.
The vertex has next siblings if it is not the root and $BP[find\_close(i) + 1]$ is an open parenthesis.
If the vertex has a next sibling, then it is at position $find\_close(i) + 1$.
The vertex has previous siblings if it is not the root and $BP[i-1]$ is a close parenthesis.
If the vertex has a previous sibling, then it is at position $find\_open(i-1)$.

\section{DFUDS}

Depth-First Unary Degree Sequence is defined recursively:
A leaf is encoded as a closing parenthesis.
An inner vertex is encoded as a $degree-1$ opening parentheses followed by a single closing parenthesis and concatenating encoding of the children.
For convenience, one opening parenthesis is prepended to make the sequence of parentheses balanced.

% TODO as lemma
Lemma.
This encoding gives balanced string of parentheses.
