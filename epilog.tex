\chapter*{Conclusion}
\addcontentsline{toc}{chapter}{Conclusion}

The aim of the thesis was to explore the existing succinct data structures for representing trees, to compares them, and possibly to improve them.
We have succeeded in all three tasks.

\bigbreak

To summarize this thesis, we have presented five ways how a static ordinal tree with $n$ vertices can be encoded using only $2n + o(n)$ bits of space.
The first three use a direct encoding in a bit string of $2n + 1$ or $2n$ bits.
The later two representations build on top of them by either replacing the index, or moving the idea of decomposition from bit strings to subtrees.
Finally, we showed a way how the data part of three different representations can be shared among them.

The representation differ by the set of operations which the support.
We proceeded from the very simples ones which do not offer much, to representations which support almost every operation.
The comparison of the supported operation can be found in the table \ref{tab:comparison}.
Our improvements, which are in more detail, described in the next section are typeset in bold.


\begin{sidewaystable}
	\centering
	\begin{tabularx}{\textwidth}{|L{1}||L{1}|L{1}|L{1}|L{1}|L{1}|}
		\hline
		Operation
		& LOUDS
		& BP
		& DFUDS
		& FF
		& TC
		\\ \hline \hline

		\anyRank{}, \newline
		\anySelect{}
		& \loAny{}
		& \preAny{}, \postAny{}, \inAny{}
		& \preAny{}, \dfudsAny{}
		& \preAny{}, \postAny{}, \dfudsAny{}, \inAny{}
		& \preAny{}, \postAny{}, \dfudsAny{}, \textbf{\inAny{} (new result)}
		\\ \hline
		
		\parent{}, \isRoot{}, \isLeaf{}
		& Yes
		& Yes
		& Yes
		& Yes
		& Yes
		\\ \hline
		
		\childAny{}, \degree{}
		& Yes
		& Yes, \childRank{}, \childSelect{}, \degree{} not-shown
		& Yes
		& \textbf{Yes (alternative)}
		& Yes
		\\ \hline
		
		\dep{}
		& \textbf{Yes, new result}
		& Yes
		& Yes, not-shown
		& Yes
		& Yes
		\\ \hline
	
		\levelAny{} (restrictable)
		& Yes
		& No
		& No
		& Only iteration (subtree)
		& \textbf{Only iteration (alternative) (subtree)}
		\\ \hline
		
		\leafAny{} (restrictable)
		& Yes, in level-order (level range)
		& Yes (subtree)
		& Yes (subtree)
		& Yes (subtree)
		& Yes (subtree)
		\\ \hline
		
		\isAncestor{}
		& No
		& Yes
		& Yes
		& Yes
		& Yes
		\\ \hline
		
		\levelAncestor{}
		& No
		& Yes, not-shown
		& Yes, not-shown
		& Yes
		& Yes
		\\ \hline
		
		\lca{}, \distance{}
		& No
		& Yes
		& Yes
		& Yes
		& Yes
		\\ \hline
		
		\hei{}, \deepestVertex{}
		& No
		& Yes
		& No
		& Yes
		& Yes
		\\ \hline
		
		\subtreeSize{}
		& No
		& Yes
		& Yes
		& Yes
		& Yes
		\\ \hline
	\end{tabularx}
	\caption{Comparison of operations supported by various representations}
	\label{tab:comparison}
\end{sidewaystable}

\bigbreak

All structures and algorithms which we present in this thesis are also described with two exceptions:
\begin{description}
	\item[Fusion tree]
	A fusion tree is required for the tiny compressed array.
	This decision can be advocated by the fact that it is only of a theoretical interest.
	
	\item[Succinct indexed dictionaty]
	The succinct indexed dictionary appears twice in our work: first in the definition of the tiny compressed array, and then also in the solution of the generalized \rmqSize{} and \rmqSelect{}.
	We did not incorporate the description of this structures as it uses technique which are too different from those which we used for representing trees.
\end{description}

We simplified some operations by not using the state of the art data structures in order to provide a description of a reasonable complexity.

\section*{Our Contribution}
\addcontentsline{toc}{section}{Our Contribution}

Our first result is the definition of the compressed array in section \ref{s:compressed-array}.
Although this technique has been widely used, we wrapped it into a structure which provides clearly defined operations.
The use of the compressed array simplified the notation in many of the presented algorithms.

Our major result the proposal of an index which supports the \dep{} and \levelAny{} operations in the LOUDS representation.
The LOUDS representation being discovered first, however the attention moved fast to the other representation since the set of operations supported by LOUDS seemed to be very restricted.
It is now the only representation which supports the operations \levelRank{} and \levelSelect{}.

For the first time, we have defined the operation \rmqSelect{} which generalizes \rmqi{} by allowing to specify the number of the desired occurrence of minimum.
Until now, only a fixed occurrence, such as the first or the last, could be found.
We also defined operations \rmqSize{}, \rmqRank{} which return the number of all occurrences of the minimum in range, or in a part of it.
We used the generalized range operations to propose an alternative algorithm for the \degree{}, \childRank{}, \childSelect{} operations in the FF representation.

We solved the operation \inRank{} and \inSelect{} for the TC representation.
Although the solution uses the same techniques as \dfudsRank{} and \dfudsSelect{}, which have already been published, the in-order operations are more complex.
This stems from the property that one vertex can be assigned multiple in-order numbers.

One more of our results is related to the TC representation.
We propose an alternative algorithms for the operations \levelFirst{} and \levelLast{} which are restricted to a subtree of a given vertex.
The query is reduced to a problem of finding a level ancestor, which we also describe at two different occasions.

The last of our results involves the universal succinct representation where we achieved two improvements.
First, we changed ownership of parenthesis in the DFUDS representation which lowered the number of runs in this case to only three.
It also simplified the algorithm captured by the look-up table generating the bits of the DFUDS representation.

The other improvement is that we analyzed carefully the constant in the definitions of a significant vertex.
We used this constant and one other to formulate inequalities which describe more precisely the behavior of the BP and DFUDS generating algorithms.
We increased the lower bound on the number of bits generated in the skinny mini-tree case from $\frac{\log n}{8}$ and $\frac{\log n}{24}$ bits for the BP and DFUDS representations to $\frac{\log n}{3}$ and $\frac{\log n}{5}$ bits.

